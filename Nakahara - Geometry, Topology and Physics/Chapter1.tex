\documentclass[a4paper]{article}

\usepackage[margin=1in]{geometry}
\usepackage{amsfonts, amsmath, mathrsfs}
\usepackage{hyperref}

\newcommand{\C}{\mathbb{C}}
\newcommand{\R}{\mathbb{R}}
\newcommand{\Q}{\mathbb{Q}}
\newcommand{\Z}{\mathbb{Z}}

\begin{document}

\begin{center}
\LARGE{Mikio Nakahara - Geometry, Topology and Physics}

\Large{Chapter 1: Quantum physics}

\large{Carter Hinsley's notes}

Rendered \today
\end{center}

\section{Analytical mechanics}

\subsection{Newtonian mechanics}

The key results of this section are due to the fact that we (for ostensibly historical reasons) say a system is \emph{conservative} when its force field $\vec{F}(\vec{x})$ may be written as a gradient $-\nabla V(\vec{x})$ of a sufficiently smooth scalar function $V$, called the \emph{potential (energy)}:
\begin{align}
    \vec{F}(\vec{x}) &= -\nabla V(\vec{x})
\end{align}
Newton's equation is
\begin{align}
    \vec{F}(\vec{x}(t)) = m\frac{d^2\vec{x}(t)}{dt^2}.
\end{align}
The \emph{(total) energy} $E$ is a quantity used for accounting for the constancy of the potential in a system; the total energy is constant (conserved) if and only if the potential $V$ is constant through time. This is by design; $E$ is chosen precisely so that this occurs:
\begin{align}
    E(t_0) = \frac{m}{2}\left(\left.\frac{d\vec{x}}{dt}\right|_{t_0}\right)^2 + V(\vec{x}(t_0)).
\end{align}
To see that $E$ is constant when $\vec{F}$ is conservative, take the time derivative:
\begin{align}
\begin{split}
    \left.\frac{dE}{dt}\right|_{t_0} &= m\left.\frac{d\vec{x}(t)}{dt}\right|_{t_0}\left.\frac{d^2\vec{x}(t)}{dt^2}\right|_{t_0} + \nabla V(\vec{x}(t_0)) \left.\frac{d\vec{x}(t)}{dt}\right|_{t_0} \\ \ \\
    &= \left[-\nabla V(\vec{x}(t_0)) + \nabla V(\vec{x}(t_0))\right]\left.\frac{d\vec{x}(t)}{dt}\right|_{t_0} = 0.
\end{split}
\end{align}

An example is given in the textbook of the nonconservative force of friction in one dimension: $F(x) = -\eta \frac{dx}{dt}$. To show that this is a nonconservative force, we may show that any potential function cannot be constant through time (note that we don't need to show the energy as defined in $(3)$ is nonconstant). By $(2)$ we have $-\eta \frac{dx}{dt} = m\frac{d^2x}{dt^2}$. Letting $\dot{x} = \frac{dx}{dt}$, this becomes $-\eta \dot{x} = m\frac{d\dot{x}}{dt}$ which is a separable equation having solution $\dot{x} = Ce^{-\frac{\eta t}{m}}$. Differentiating and using $(2)$ again, we obtain $F(x) = -\eta Ce^{-\frac{\eta t}{m}}$ so that $(1)$ yields $\frac{dV}{dx} = \eta Ce^{-\frac{\eta t}{m}}$. Integrating over $x$, we obtain $V(x) = x\eta Ce^{-\frac{\eta t}{m}} + x_0$. The potential is therefore nonconstant through time.

\end{document}