\documentclass[a4paper]{article}

\usepackage[margin=1in]{geometry}
\usepackage{amsfonts, amsmath, mathrsfs}

\newcommand{\C}{\mathbb{C}}
\newcommand{\R}{\mathbb{R}}
\newcommand{\Q}{\mathbb{Q}}
\newcommand{\Z}{\mathbb{Z}}

\begin{document}

\begin{center}
\LARGE{Walter Rudin - Principles of Mathematical Analysis}

\Large{Chapter 7: Sequences and series of functions}

\large{Carter Hinsley's notes}

Rendered \today
\end{center}

\section{Discussion of main problem}

\textbf{Definition 7.1.} Let $\{f_n\}$ be a sequence of functions $E \to \C$, where the image sequence $\{f_n(x)\}$ converges for every $x \in E$. We can then define the \emph{limit function} $\displaystyle f(x) := \lim_{n\to\infty} f_n(x)$ as a map $E \to \C$. Likewise, if $\sum f_n(x)$ converges for every $x \in E$, we can define the \emph{sum} $\displaystyle f(x) := \sum f_n(x)$ of the series $\sum f_n$.

We care to determine what properties of functions in the sequence $\{f_n\}$ are preserved under these limit operations (i.e., taking the limit and sum functions).

In particular, the statement that the limit of a sequence of continuous functions is continuous is equivalent to the interchange of limits $\displaystyle \lim_{t \to x} \lim_{n \to \infty} f_n(t) = \lim_{n \to \infty} \lim_{t \to x} f_n(t)$. An example of when this is not true is $\displaystyle 1 = \lim_{n \to \infty} \lim_{m \to \infty} \frac{m}{m + n} \neq \lim_{m \to \infty} \lim_{n \to \infty} \frac{m}{m + n} = 0$. Further examples show that sum functions of sequences may be discontinuous, that limits of sequences of continuous functions may be \emph{everywhere} discontinuous, that derivatives of limits of sequences of differentiable functions may disagree with the derivatives of the sequential functions, and that the integrals of limits of sequences of integrable functions may disagree with the integrals of the sequential functions.

\section{Uniform convergence}

A new, stronger (than in Definition 7.1) notion of convergence is needed to cause limits of sequences of functions to preserve analytical properties.

\textbf{Definition 7.7.} A sequence $\{f_n\}$ of functions $E \to \C$ converges \emph{uniformly} on $E$ to a function $f : E \to \C$ if for every $\varepsilon > 0$ there exists $N \in \Z^+$ such that $n \geq N$ implies $|f_n(x) - f(x)| < \varepsilon$ for all $x \in E$.

\textbf{Theorem (Cauchy criterion for uniform convergence).} A sequence $\{f_n\}$ of functions $E \to \C$ converges uniformly on $E$ iff for every $\varepsilon > 0$ there exists $N \in \Z^+$ such that $m, n \geq N$ implies $|f_n(x) - f_m(x)| < \varepsilon$.

\textbf{Theorem 7.9.} Suppose $\displaystyle \lim_{n \to \infty} f_n(x) = f(x)$ for all $x \in E$. Let $\displaystyle M_n = \sup_{x \in E} |f_n(x) - f(x)|$. Then $f_n \to f$ uniformly on $E$ iff $M_n \to 0$ as $n \to \infty$.

\textbf{Remark} The above is pretty much indistinguishable from Definition 7.7. I'm not sure why Rudin thought it pertinent to add this ``Theorem.''

A test for uniform convergence of series due to Weierstrass follows.

\textbf{Theorem 7.10.} Suppose $\{f_n\}$ is a sequence of functions $E \to \C$ and $\{M_n\} \subset \R$ is defined so that $|f_n(x)| \leq M_n$ for all $x \in E, n \in \Z^+$. Then $\sum f_n$ converges uniformly on $E$ if $\sum M_n$ converges. One can think of this as assigning a bound $\pm M_n$ on the image set of each summand function $f_n$ and checking if the resultant bound series absolutely converges; this technique is actually rather obvious.

\section{Uniform convergence and continuity}

In this section we check whether limits of uniformly convergent sequences of functions are themselves continuous and we also show that the space of complex, continuous, bounded functions on a metric space form a new metric space.

\textbf{Theorem 7.11.} Suppose $f_n \to f$ uniformly on a set $E$ in a metric space. Let $x$ be a limit point of $E$. Then $\displaystyle \lim_{t \to x} \lim_{n \to \infty} f_n(t) = \lim_{n \to \infty} \lim_{t \to x} f_n(t)$.

\textbf{Theorem 7.12.} Suppose $f_n \to f$ uniformly on $E$ where each $f_n$ is continuous. Then $f$ is continuous on $E$. (This is immediate from Theorem 7.11.)

We would like to know when then converse is true; the following theorem can help.

\textbf{Theorem 7.13.} Suppose $K$ is compact and the following three conditions are met:

\begin{itemize}
    \item $\{f_n\}$ is a sequence of continuous functions $K \to \C$,
    \item $f_n \to f$ pointwise where $f$ is a continuous function $K \to \C$,
    \item $f_n(x) \geq f_{n+1}(x)$ for all $x \in K, n \in \Z^+$ (i.e., $\{f_n(x)\}$ is a monotone decreasing sequence for all $x \in K$).
\end{itemize}

Then $f_n \to f$ uniformly on $K$.

The following results demonstrate that the complex, continuous, bounded functions on a metric space themselves form a new metric space.

\textbf{Definition 7.14.} Let $X$ be a metric space. $\mathscr{C}$ is the set of all complex-valued, continuous, bounded functions with domain $X$. This is a normed space: for all $f \in \mathscr{C}(X)$ we have the \emph{supremum norm} $\displaystyle \|f\| = \sup_{x \in X} |f(x)|$. The metric is then the usual $d_{\mathscr{C}(X)}(f, g) = \|f - g\|$. Closed subsets of $\mathscr{C}(X)$ are called \emph{uniformly closed}. The closure of a set $\mathscr{A} \subseteq \mathscr{C}(X)$ is called its \emph{uniform closure}.

\textbf{Theorem 7.15.} The above metric makes $\mathscr{C}(X)$ into a \textbf{complete} metric space.

\section{Uniform convergence and integration}

The results of this section establish the integrability and commutativity of limits with integrals under assumption of uniform convergence.

\textbf{Theorem 7.16.} Let $\alpha : [a, b] \to \R$ be monotonically increasing. Suppose $f_n \in \mathscr{R}(\alpha)$ on $[a, b]$ for all $n \in \Z^+$ and suppose $f_n \to f$ uniformly on $[a, b]$. Then $f \in \mathscr{R}(\alpha)$ on $[a, b]$, and $\displaystyle \int_a^b f\ d\alpha = \lim_{n \to \infty} \int_a^b f_n\ d\alpha$.

\textbf{Corollary.} If $f_n \in \mathscr{R}(\alpha)$ on $[a, b]$ and if $f(x) = \sum f_n(x)$ for all $x \in [a, b]$, the series converging uniformly on $[a, b]$, then $\displaystyle \int_a^b f\ d\alpha = \sum \int_a^b f_n\ d\alpha$.

\section{Uniform convergence and differentiation}

\section{Equicontinuous families of functions}

\section{The Stone-Weierstrass theorem}

\end{document}