\documentclass[a4paper]{article}

\usepackage[margin=1in]{geometry}
\usepackage{amsfonts, amsmath, mathrsfs}
\usepackage{hyperref}

\newcommand{\C}{\mathbb{C}}
\newcommand{\R}{\mathbb{R}}
\newcommand{\Q}{\mathbb{Q}}
\newcommand{\Z}{\mathbb{Z}}

\begin{document}

\begin{center}
\LARGE{Shlomo Sternberg - Lectures on Differential Geometry}

\Large{Chapter 1: Algebraic preliminaries}

\large{Carter Hinsley's notes}

Rendered \today
\end{center}

In passing from differentiable objects to their infinitesimal analogues, we effectively linearize structures and study the local case. All of these algorithms fall under the heading \emph{multilinear algebra}. We consider only vector spaces over $\R$.

\section{Tensor products of vector spaces}

Let $V, W$ be vector spaces with dimension $m, n$ respectively. Then $\hom(V, W)$ is a vector space of dimension $mn$. Denote by $V^*$ the dual space of $V$. Define the map $\varphi : V \times W \to \hom(V^*, W)$ as follows: $\varphi(v, w)(v^*) = \langle v, v^* \rangle w = v^*(v) w$. That is, $\varphi(v, w)$ accepts a linear functional acting on $v$ to produce a scalar coefficient for $w$.

\textbf{Exercise 1.3.} Let $v_1, \ldots, v_m$ be a basis of $V$ and $w_1, \ldots, w_n$ a basis of $W$. Show that $\{\varphi(v_i, w_j)\},\ i = 1, \ldots, m,\ j = 1, \ldots, n$ is a basis of $\hom(V^*, W)$.

\emph{Proof.} $\hom(V^*, W)$ is an $mn$-dimensional vector space; we need only to show that the $\varphi(v_i, w_j)$ are linearly independent. To do this, note that whenever $\sum_{i=1}^m a_iv_i = 0$ ($a_i \in \R$) or $\sum_{j=1}^n b_jw_j = 0$ ($b_j \in \R$) we will have $a_i = 0$ for all $1 \leq i \leq m$ or $b_j = 0$ for all $1 \leq j \leq n$ respectively, because the $v_i$ and $w_j$ form bases of their respective spaces. Similarly, take a linear combination $\sum_{i, j} c_{ij} \varphi(v_i, w_j) = 0$. We must now show that this implies $c_{ij} = 0$ for all $i, j$. Pick out $v^* \in V^*$ arbitrarily; we shall then have $0 = \sum_{i, j} c_{ij}\varphi(v_i, w_j)(v^*) = \sum_{i, j} c_{ij}\langle v_i, v^* \rangle w_j$. Fixing any $j$ we obtain $0 = \sum_i c_{ij} \langle v_i, v^* \rangle$. Taking the dual basis $\{v_i^*\}$ to $\{v_i\}$ we have $v^* = \sum_i \alpha_iv_i^*$ so that $0 = \sum_i c_{ij} \langle v_i, \sum_{k=1}^m \alpha_k v_k^* \rangle = \sum_i c_{ij} \sum_{k=1}^m \alpha_k \langle v_i, v_k^* \rangle$, where each $\alpha_k$ is arbitrary and $\langle v_i, v_i^* \rangle$ is strictly positive. Hence $c_{ij} = 0$ for all $i, j$. $\square$

\end{document}