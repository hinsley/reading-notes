\documentclass[a4paper]{article}

\usepackage[margin=1in]{geometry}
\usepackage{amsfonts, amsmath, mathrsfs}
\usepackage{hyperref}

\newcommand{\C}{\mathbb{C}}
\newcommand{\R}{\mathbb{R}}
\newcommand{\Q}{\mathbb{Q}}
\newcommand{\Z}{\mathbb{Z}}

\begin{document}

\begin{center}
\LARGE{Shlomo Sternberg - Lectures on Differential Geometry}

\Large{Chapter 1: Algebraic preliminaries}

\large{Carter Hinsley's notes}

Rendered \today
\end{center}

In passing from differentiable objects to their infinitesimal analogues, we effectively linearize structures and study the local case. All of these algorithms fall under the heading \emph{multilinear algebra}. We consider only vector spaces over $\R$.

\section{Tensor products of vector spaces}

Let $V, W$ be vector spaces with dimension $m, n$ respectively. Then $\hom(V, W)$ is a vector space of dimension $mn$. Denote by $V^*$ the dual space of $V$. Define the map $\varphi : V \times W \to \hom(V^*, W)$ as follows: $\varphi(v, w)(v^*) = \langle v, v^* \rangle w = v^*(v) w$. That is, $\varphi(v, w)$ accepts a linear functional acting on $v$ to produce a scalar coefficient for $w$.

\textbf{Exercise 1.3.} Let $v_1, \ldots, v_m$ be a basis of $V$ and $w_1, \ldots, w_n$ a basis of $W$. Show that $\{\varphi(v_i, w_j)\},\ i = 1, \ldots, m,\ j = 1, \ldots, n$ is a basis of $\hom(V^*, W)$.

\emph{Proof.} $\hom(V^*, W)$ is an $mn$-dimensional vector space; we need only to show that the $\varphi(v_i, w_j)$ are linearly independent. 

\end{document}