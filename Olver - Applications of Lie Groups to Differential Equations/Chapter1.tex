\documentclass[a4paper]{article}

\usepackage[margin=1in]{geometry}
\usepackage{amsfonts, amsmath, mathrsfs}
\usepackage{hyperref}

\newcommand{\C}{\mathbb{C}}
\newcommand{\R}{\mathbb{R}}
\newcommand{\Q}{\mathbb{Q}}
\newcommand{\Z}{\mathbb{Z}}

\begin{document}

\begin{center}
\LARGE{Peter Olver - Applications of Lie Groups to Differential Equations}

\Large{Chapter 1: Introduction to Lie groups}

\large{Carter Hinsley's notes}

Rendered \today
\end{center}

\section{Manifolds}

The usual notation and background goes on here. Interestingly, Olver states that the coordinate charts on a manifold yield a basis for a Hausdorff topology by the use of the subspace topology on the image of the corresponding coordinate functions. Another interesting thing to note is that Olver doesn't say the coordinate functions need to be diffeomorphisms; they must merely be injective. It is the transition maps that must be diffeomorphisms.

In example 1.4, Olver claims the torus $T^2 = S^1 \times S^1$ can be covered by three charts. Why not two? I should think that you could just take two annuli and cover the whole thing. But, well, if you could do that, then $S^1$ could be parametrized by a \emph{single} chart rather than two. It might be worth verifying that two such charts do not satisfy all the criteria for charts. Images of charts should be simply connected; a friend says this criterion means charts (as maps) should be ``simply supported.'' But the problem warrants deeper investigation.

I will try to construct a collection of charts for a torus using only two charts. Let $(\theta, \varphi)$ describe a point on the torus in terms of two angles, where $0 \leq \theta, \varphi < 2\pi$. Let $U_\alpha = \{(\theta, \varphi) \mid \theta \neq 0\}$ and $U_\beta = \{(\theta, \varphi) \mid \theta \neq \pi\}$. Then $U_\alpha \cap U_\beta = \{(\theta, \varphi) \mid \theta \notin \{0, \pi\}\}$. Taking $\chi_\alpha(\theta, \varphi) = (\theta, \varphi)$ to be a map to polar coordinates for $\R^2$ where radius is specified by $\theta$ we get an injective coordinate map. Similarly, $\chi_\beta$ can be given as a piecewise map into polar coordinates $\chi_\beta(\theta, \varphi) = \left\{ \begin{array}{ll} (\pi - \theta, \varphi) & \text{if } \theta < \pi, \\ (3\pi - \theta, \varphi) & \text{if } \theta > \pi. \end{array} \right.$ We now must check that $\chi_\alpha \circ \chi_\beta^{-1}$ and $\chi_\beta \circ \chi_\alpha^{-1}$ are smooth. To do this, we find the inverses:
\begin{align}
    \chi_\alpha^{-1}(r, \varphi) &: \begin{array}{rcl} \{(r, \varphi) \mid 0 < r < 2\pi\} & \to & U_\alpha \\ (r, \varphi) & \mapsto & (r, \varphi) \end{array}
\end{align}
\begin{align}
    \chi_\beta^{-1}(r, \varphi) &: \begin{array}{rcl} \{(r, \varphi) \mid 0 < r < 2\pi\} & \to & U_\beta \\ (r, \varphi) & \mapsto & \left\{ \begin{array}{ll} (\pi - r, \varphi) & \text{if } r \leq \pi \\ (3\pi - r, \varphi) & \text{if } r > \pi \end{array} \right. \end{array}
\end{align}
The transition maps are then
\begin{align}
    \chi_\beta \circ \chi_\alpha^{-1} : \begin{array}{rcl} \chi_\alpha(U_\alpha \cap U_\beta) & \to & \chi_\beta(U_\alpha \cap U_\beta) \\ (r, \varphi) & \mapsto & \left\{ \begin{array}{ll} (\pi - r, \varphi) & \text{if } r \leq \pi \\ (3\pi - r, \varphi) & \text{if } r > \pi \end{array} \right. \end{array}
\end{align}
\begin{align}
    \chi_\alpha \circ \chi_\beta^{-1} : \begin{array}{rcl} \chi_\beta(U_\alpha \cap U_\beta) & \to & \chi_\alpha(U_\alpha \cap U_\beta) \\ (r, \varphi) & \mapsto & \left\{ \begin{array}{ll} (\pi - r, \varphi) & \text{if } r \leq \pi \\ (3\pi - r, \varphi) & \text{if } r > \pi. \end{array} \right. \end{array}
\end{align}

These maps look discontinuous at first glance, but we need to check against their domains to be certain. Note that $\chi_\alpha(U_\alpha \cap U_\beta) = \chi_\beta(U_\beta \cap U_\alpha) = \{(r, \varphi) \mid 0 < |r - \pi| < \pi, 0 \leq \varphi < 2\pi\}$. So the transition maps can be rewritten
\begin{align}
    \chi_\beta \circ \chi_\alpha^{-1} : \begin{array}{rcl} \chi_\alpha(U_\alpha \cap U_\beta) & \to & \chi_\beta(U_\alpha \cap U_\beta) \\ (r, \varphi) & \mapsto & \left\{ \begin{array}{ll} (\pi - r, \varphi) & \text{if } r < \pi \\ (3\pi - r, \varphi) & \text{if } r > \pi. \end{array} \right. \end{array}
\end{align}
\begin{align}
    \chi_\alpha \circ \chi_\beta^{-1} : \begin{array}{rcl} \chi_\beta(U_\alpha \cap U_\beta) & \to & \chi_\alpha(U_\alpha \cap U_\beta) \\ (r, \varphi) & \mapsto & \left\{ \begin{array}{ll} (\pi - r, \varphi) & \text{if } r < \pi \\ (3\pi - r, \varphi) & \text{if } r > \pi \end{array} \right. \end{array}
\end{align}
and it is then clear that these are smooth: they are linear over connected components.

The reason that Olver suggests the use of three charts is that he aims to only use simply connected charts; if we do not make this restriction, we can easily cover the torus with just two charts as above. See \href{https://math.stackexchange.com/a/2898114/921952}{this Math Exchange post} for a similar discussion. I have been told by someone that we generally require simply connected charts by definition in most texts.

\subsection{Change of coordinates}

You can always extend a collection of charts via redundant charts \& coordinate functions so long as these charts are admissible; i.e., the resultant transition maps are all smooth. This permits the choice between multiple coordinate systems for a chart on a manifold. Also, any composition $\psi \circ \chi_\alpha$ --- where $\chi_\alpha : U_\alpha \to V_\alpha \subseteq \R^n$ is a local coordinate function and $\psi : V_\alpha \to \tilde{V}_\alpha \subseteq \R^n$ is an arbitrary diffeomorphism, called a \emph{change of coordinates} --- yields a new coordinate system for a given chart. Any property of the manifold $M$ under the coordinates $\chi_\alpha$ or $\psi \circ \chi_\alpha$ must be preserved. We therefore never need to make a restriction to a particular coordinate system when talking about properties of manifolds other than for purposes of computation; all results on a manifold are intrinsically coordinate-free in nature.

A maximal collection of (compatible) charts is called an \emph{atlas}. Atlases behave as do charted covers of manifolds, but since manifolds come associated with countable collections of charts, an atlas is always merely implicit in the specification of a manifold. But due to the mutual compatibility of any redundant charts, the atlas is uniquely determined for any manifold. Most of the time we'll talk about points on a manifold in terms of local coordinates without any description of a particular coordinate system.

\subsection{Maps between manifolds}

A map between manifolds $F : M \to N$ is said to be smooth if it is smooth in every pair of corresponding local coordinate systems. 

\subsection{The maximal rank condition}

\textbf{Definition 1.7.} Let $F : M \to N$ be a smooth map from an $m$-dimensional manifold to an $n$-dimensional manifold. The \emph{rank} of $F$ at $x \in M$ is the rank of the $m \times n$ Jacobian matrix $\left(\frac{\partial F^i}{\partial x^j}\right)$ at $x$. $F$ is of \emph{maximal rank} on a subset $S \subseteq M$ if for each $x \in S$ the rank of $F$ is maximized over $M$ (i.e., $F$ has rank equal to $\min\{m, n\}$).

\textbf{Theorem 1.8.} Let $F : M \to N$ be of maximal rank at $x_0 \in M$. Then there are local coordinates $x = (x^1, \ldots, x^m)$ near $x_0$ and $y = (y^1, \ldots, y^n)$ near $y_0 = F(x_0)$ such that in these coordinates $F$ has the simple form $y = (x^1, \ldots, x^m, 0, \ldots, 0)$ if $m < n$ or $y = (x^1, \ldots, x^n)$ if $m \geq n$.

The above two results, as far as I can presently see, seem to suggest that most of the time we should be able to think of smooth maps between manifolds as locally being projections or immersions.

\subsection{Submanifolds}

Submanifolds are subsets of manifolds which are themselves manifolds. There are two ways to describe submanifolds: either by implicit representation by the zeroes of some smooth function or by explicit parametrization by a collection of charts.

\textbf{Definition 1.9.} Let $M$ be a smooth manifold. A \emph{submanifold} of $M$ is a subset $N \subseteq M$ together with a smooth bijection $\phi : \tilde{N} \hookrightarrow N$ satisfying the maximal rank condition everywhere. We call $\tilde{N}$, another manifold, the \emph{parameter space} of $N$. $\phi$ is called an \emph{immersion}. Also, $\dim \tilde{N} = \dim N \leq \dim M$.

\textbf{Remark.} Immersed submanifolds do not have singularities. This is due to the maximal rank condition.

It's not quite clear to me why a parameter space must be chosen for a given submanifold. It's also not clear to me why it need not be diffeomorphic to the submanifold. I suspect the answer to each of these questions is related and may be apparent from going over the examples in this section. (It turns out this is true; see example (c).)

\textbf{Example 1.10.} In each of the following examples, $\tilde{N} = \R$ and $\phi : \R \to M$ with $\phi(\R) = N$; $N$ is a submanifold of $M$ parametrized by $\tilde{N}$.

(a) Let $M = \R^3$ and $\phi(t) = (\cos t, \sin t, t)$. This is the standard helix parametrization. This is a regular curve as $\dot{\phi}(t) = (-\sin t, \cos t, 1)$. So it satisfies the maximal rank condition everywhere.

(b) Let $M = \R^2$ and $\phi(t) = ((1 + e^{-t})\cos t, (1 + e^{-t})\sin t)$. This is a logarithmic spiral towards the unit circle.

(c) Let $M = \R^2$ and $\hat\phi(t) = (\sin t, 2\sin(2t))$. This is a figure eight, which has self-intersections whenever $t = k\pi, k \in \Z$. This does not constitute a submanifold of $M$ as the maximal rank condition is not satisfied everywhere. However, $\phi(t) = (\sin(2\arctan t), 2\sin(4\arctan t))$ yields a submanifold of $M$, $N = \phi(\R)$ (consider the range of $\arctan$). This is the ``same'' subspace, but without multiplicities; this one is actually a submanifold. Moreover, we could have taken the alternative immersion $\tilde\phi(t) = (-\sin(2\arctan t), 2\sin(4\arctan t))$. This has the same image but $\phi \circ \tilde\phi^{-1} : \R \to \R$ is not continuous at zero. The point of this example is to show that specifying a submanifold only by a subset of a manifold is ambiguous; it is required that we specify the immersion as well. This necessitates the use of a parameter space as well so that the immersion has a domain.

Example (d) given is not very insightful; it's just a spiral going around a 2-torus. The interesting thing is that for irrational choices of period, we end up with a curve on the torus whose closure is the whole toroidal surface.

\end{document}