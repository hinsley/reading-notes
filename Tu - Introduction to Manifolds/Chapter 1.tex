\documentclass[a4paper]{article}

\usepackage[margin=1in]{geometry}
\usepackage{amsfonts, amsmath, mathrsfs}
\usepackage{hyperref}

\newcommand{\C}{\mathbb{C}}
\newcommand{\R}{\mathbb{R}}
\newcommand{\Q}{\mathbb{Q}}
\newcommand{\Z}{\mathbb{Z}}

\begin{document}

\begin{center}
\LARGE{Loring Tu - Introduction to Manifolds}

\Large{Chapter 1: Euclidean spaces}

\large{Carter Hinsley's notes}

Rendered \today
\end{center}

\section{Smooth functions on a Euclidean space}

\subsection{$C^\infty$ versus analytic functions}

Analytic functions are equal to their Taylor series expansions in some open neighborhood of the point at which the expansion is taken. Meanwhile, there is a $C^\infty$ function which does not agree on any neighborhood of $0$ with its Maclaurin series expansion (which is identically zero on $\R$):

\begin{align}
    f(x) = \left\{\begin{matrix} e^{-1/x} & \text{for } x > 0, \\ 0 & \text{for } x \leq 0. \end{matrix}\right.
\end{align}

\textbf{Lemma 1.4 (Taylor's theorem with remainder).} Let $f$ be a $C^\infty$ function on a star-shaped (with respect to a point $p = (p^1, \ldots, p^n) \in U$) subset $U \subseteq \R^n$. Then there are functions $g_1(x), \ldots, g_n(x) \in C^\infty(U)$ such that $g_i(p) = \frac{\partial f}{\partial x^i}(p)$ and

\begin{align}
\begin{split}
    f(x) &= f(p) + \sum_{i=1}^n (x^i - p^i)g_i(x).
\end{split}
\end{align}

\section*{Problems}

\subsection*{1.1. A function that is $C^2$ but not $C^3$}

Let $g : \R \to \R$ be the function in Example 1.2(iii). Show that the function $h(x) = \int_0^x g(t)\ dt$ is $C^2$ but not $C^3$ at $x = 0$.

\textbf{Proof.} Recall from Example 1.2(iii) that $g'(x) = f(x)$ while $f'(0)$ is undefined. It is not difficult to check that $f(x)$ is continuously differentiable at $x = 0$ but $f'(x)$ is not defined at $x = 0$. $\square$

\section{Tangent vectors in $\R^n$ as derivations}

A vector in $\R^3$ at a point $p$ is usually represented as a $3\times1$ "column" matrix or as an arrow emanating from $p$. A secant plane to a surface in $\R^3$ is a plane determined by three points on the surface. Taking these three points to approach $p$ in the limit, we obtain the \emph{tangent plane} to the surface at $p$. This representation of a tangent plane unfortunately only works when the surface is embedded in Euclidean space. We would rather have a representation intrinsic to the surface (or manifold).

\subsection{The directional derivative}

We write elements of the tangent space $T_p\R^n$ as $\langle v^1, \ldots, v^n \rangle$. The line through a point $p = (p^1, \ldots, p^n)$ with direction $v = \langle v^1, \ldots, v^n \rangle$ in $\R^n$ has parametrization $c(t) = (p^1 + tv^1, \ldots, p^n + tv^n)$. If $f$ is $C^\infty$ in a neighborhood of $p$ in $\R^n$ and $v \in T_p\R^n$, the \emph{directional derivative} of $f$ in the direction $v$ at $p$ is 

\begin{align}
    \displaystyle D_vf &= \lim_{t \to 0} \frac{f(c(t)) - f(p)}{t} = \left.\frac{d}{dt}\right|_{t=0} f(c(t)) = \sum_{i=1}^n v^i \frac{\partial f}{\partial x^i}(p).
\end{align}

We may unambiguously write

\begin{align}
    D_v = \sum v^i \left.\frac{\partial}{\partial x^i}\right|_p.
\end{align}

It's clear that $D : v \mapsto D_v$ creates a correspondence between tangent vectors and differential operators on functions so that we can study the latter.

\subsection{Germs of functions}

If two functions $f$ and $g$ agree on some neighborhood of a point $p$, then $Df$ and $Dg$, being maps $T_p\R^n \to \R$, are identical. Let $f : U \to \R$ be a $C^\infty$ function where $U$ is an open neighborhood of $p$. The pair $(f, U)$ is equivalent to $(g, V)$ if there is an open neighborhood $W \subseteq U \cap V$ of $p$ wherein $f|_W = g|_W$. The equivalence class of $(f, U)$ is called the \emph{germ} of $f$ at $p$; the set of all germs of $C^\infty$ functions on $\R^n$ at $p$ is denoted $C_p^\infty(\R^n)$ or $C_p^\infty$.

\textbf{Example.} (From the \href{https://ltu.pages.tufts.edu/doc/mf2_master_errata2.pdf}{errata}) The set $C^\infty$(U) of all $C^\infty$ functions on an open set $U \subseteq \R^n$ is a $\R$-algebra.

A $k$-algebra is a ring $R$ equipped with a scalar action from a field $k$ satisfying homogeneity: $c(ab) = (ca)b = a(cb)$ where $a, b \in R$ and $c \in k$. The usual linearity terminology for maps applies to algebras. Algebra homomorphisms are linear maps that preserve the algebra multiplication (i.e., the multiplication of the underlying ring).

\textbf{Example.} The addition and multiplication of functions render $C_p^\infty$ an $\R$-algebra.

\subsection{Derivations at a point}

Any linear map $D : C_p^\infty \to \R$ satisfying the Leibniz rule $D_v(fg) = (D_vf)g(p) + f(p)D_vg$ (in other words, the product rule) is called a \emph{derivation at $p$} or a \emph{point-derivation} of $C_p^\infty$. The set of all derivations at $p$ is denoted by $\mathfrak{D}_p(\R^n)$; this itself is a vector space over $\R$.

\textbf{Theorem 2.2.} The linear map
\begin{align}
\begin{split}
    \phi : T_p\R^n &\to \mathfrak{D}_p(\R^n) \\
    v &\mapsto D_v = \sum v^i \left.\frac{\partial}{\partial x^i}\right|_p
\end{split}
\end{align}
is a vector space isomorphism.

\textbf{Proof.} Injectivity is easy to see. As for surjectivity, let $D \in \mathfrak{D}_p(\R^n)$ and let $(f, U)$ be a representative of a germ in $C_p^\infty$, where $U$ is an open ball. By Taylor's theorem with remainder, we have $f(x) = f(p) + \sum (x^i - p^i)g_i(x)$ and $g_i(p) = \frac{\partial f}{\partial x^i}(p)$. Applying $D$ to both sides, Leibniz rule and vanishing constants yield

\begin{align}
\begin{split}
    Df(x) &= \sum (Dx^i)g_i(p) + \sum (p^i - p^i) Dg_i(x) \\ \ \\
    &= \sum (Dx^i) \frac{\partial f}{\partial x^i}(p).
\end{split}
\end{align}

Hence $D = D_v$ where $v = \langle Dx^1, \ldots, Dx^n \rangle$. $\square$

We have just shown that there is a one-to-one correspondence between the tangent vectors at $p$ and the derivations at $p$; in particular, the point-derivations are precisely the directional derivatives. This vector space $\mathfrak{D}_p(\R^n)$ may be generalized to smooth manifolds, so we will keep its structure in mind as we move onward.

Theorem 2.2 admits a correspondence between bases:

\begin{align}
    e^i \overset{\phi}{\longmapsto} \left.\frac{\partial}{\partial x^i}\right|_p.
\end{align}

From now on, the author ``will make this identification and write a tangent vector $v = \langle v^1, \ldots, v^n \rangle = \sum v^ie_i$ as $v = \sum v^i \left.\frac{\partial}{\partial x^i}\right|_p$''. We instead use Einstein summative notation and write a tangent vector $v = v^ie_i$ as $v = v^i \partial_{x^i}|_p$.

\section{Vector fields}

A vector field $X$ on an open subset $U$ of $\R^n$ is a function $p \mapsto X_p$ where $p \in U$ and $X_p \in T_p\R^n$. In general, a vector field is a function taking points on some manifold to tangent vectors in the tangent spaces of those points.

As $T_p\R^n$ has basis $\{\partial_{x^i}|_p\}$, we can write $X_p = a^i(p)\partial_{x^i}|_p$, where $p \in U$ and $a^i(p) \in \R$. With some abuse of notation we can write $X = a^i \partial_{x^i}$. The vector field $X$ is said to be $C^\infty$ on $U$ if the coefficient functions $a^i$ are all $C^\infty$ on $U$. We can write vector fields on $U$ with column vectors of $C^\infty$ functions on $U$ (this generalizes to charts):
\begin{align}
    X = a^i\partial_{x^i} \longleftrightarrow \begin{bmatrix} a^1 \\ \vdots \\ a^n \end{bmatrix}.
\end{align}

We denote by $C^\infty(U)$ or $\mathcal{F}(U)$ the ring of $C^\infty$ functions on an open set $U$. The operations are pointwise. We denote by $\mathfrak{X}(U)$ the set of all $C^\infty$ vector fields on $U$. This is both a vector space over $\R$ and a $C^\infty(U)$-module.

\subsection{Vector fields as derivations}

When $X$ is a $C^\infty$ vector field on $U$ open in $\R^n$ and $f \in C^\infty(U)$ we can speak of the function $Xf \in C^\infty(U)$, where $(Xf)(p) = X_pf$ for any $p \in U$. Hence ``a $C^\infty$ vector field $X$ gives rise to an $\R$-linear map'':
\begin{align}
\begin{split}
    C^\infty(U) &\to C^\infty(U), \\
    f &\mapsto Xf.
\end{split}
\end{align}
$\R$-linearity here comes from the equation $X = a^i\partial_{x^i}$, though the author slightly obscures this fact.

\textbf{Proposition 2.6 (Leibniz rule for a vector field).} Let $X$ be a $C^\infty$ vector field on an open subset $U$ of $\R^n$. Let $f, g \in C^\infty(U)$. Then $X(fg)$ satisfies Leibniz rule.

If $A$ is a $k$-algebra, a \emph{derivation} of $A$ is a $k$-linear map $D : A \to A$ satisfying Leibniz rule. The set of all derivations of $A$, $\text{Der}(A)$, is itself a vector space. The author now says that ``as noted above, a $C^\infty$ vector field on an open set $U$ gives rise to a derivation of the algebra $C^\infty(U)$.'' $X$ is indeed that derivation.

Additional commentary is warranted here. We have a mental image of a vector field being an attachment of ``hairs'' to an open set $U$ (in general, a manifold). But we can precisely identify the individual tangent vectors in these vector fields with directional derivatives, which form endomorphisms on $C^\infty(U)$ because the derivative of a $C^\infty$ function is still $C^\infty$. Moreover, because of the surjectivity of the map in Theorem 2.2 (stating that tangent spaces are linearly isomorphic to point-derivation spaces), we can go as far as to identify vector fields with derivations at large.

The objects we have described thus far are:
\begin{itemize}
    \item $\R^n$: $n$-dimensional Euclidean space.
    \item $T_p\R^n$: The tangent space to a point $p \in \R^n$.
    \item $(f, U)$: A representative of a germ. Germs are equivalence classes of functions having the same derivative structure at a given point $p$.
    \item $C_p^\infty(\R^n)$: The $\R$-algebra of all germs of $C^\infty$ functions on $\R^n$ at a point $p \in \R^n$.
    \item $C^\infty(U)$: The $\R$-algebra of all $C^\infty$ functions on an open set $U \subseteq \R^n$.
    \item $D : C_p^\infty \to \R$: If such a map satisfies Leibniz rule, this is called a derivation at $p$ or point-derivation of $C_p^\infty$.
    \item $\mathfrak{D}_p(\R^n)$: The set of all derivations at $p$. This is a vector space over $\R$.
    \item $X : U \to T_pU$: A vector field on an open set $U \subseteq \R^n$. Abusing notation somewhat, we can write $X = a^i\partial_{x^i}$ and get $X \in \text{End}(C^\infty(U))$.
    \item $\mathfrak{X}(U)$: The set of all $C^\infty$ vector fields on $U$. This is both a vector space over $\R$ and a $C^\infty(U)$-module.
    \item $X_p \in T_pU$: Implemented as $X_p = a^i(p)\partial_{x^i}|_p$, this is a tangent vector at a point $p \in U$.
    \item $X_pf \in \R$: Directional derivative of $f$ at the point $p$ in the direction of the tangent vector $X_p$.
    \item $Xf \in C^\infty(U)$: A vector field (see the entry for $X : U \to T_pU$: ``Abusing notation\ldots''). $(Xf)(p) = X_pf$.
\end{itemize}

\end{document}