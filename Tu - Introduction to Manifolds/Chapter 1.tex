\documentclass[a4paper]{article}

\usepackage[margin=1in]{geometry}
\usepackage{amsfonts, amsmath, mathrsfs}
\usepackage{hyperref}

\newcommand{\C}{\mathbb{C}}
\newcommand{\R}{\mathbb{R}}
\newcommand{\Q}{\mathbb{Q}}
\newcommand{\Z}{\mathbb{Z}}

\begin{document}

\begin{center}
\LARGE{Loring Tu - Introduction to Manifolds}

\Large{Chapter 1: Euclidean spaces}

\large{Carter Hinsley's notes}

Rendered \today
\end{center}

\section{Smooth functions on a Euclidean space}

\subsection{$C^\infty$ versus analytic functions}

Analytic functions are equal to their Taylor series expansions in some open neighborhood of the point at which the expansion is taken. Meanwhile, there is a $C^\infty$ function which does not agree on any neighborhood of $0$ with its Maclaurin series expansion (which is identically zero on $\R$):

\begin{align}
    f(x) = \left\{\begin{matrix} e^{-1/x} & \text{for } x > 0, \\ 0 & \text{for } x \leq 0. \end{matrix}\right.
\end{align}

\textbf{Lemma 1.4 (Taylor's theorem with remainder).} Let $f$ be a $C^\infty$ function on a star-shaped (with respect to a point $p = (p^1, \ldots, p^n) \in U$) subset $U \subseteq \R^n$. Then there are functions $\frac{\partial f}{\partial x^1}(x), \ldots, \frac{\partial f}{\partial x^n}(x) \in C^\infty(U)$ such that

\begin{align}
\begin{split}
    f(x) &= f(p) + \sum_{i=1}^n (x^i - p^i)\frac{\partial f}{\partial x^i}(p).
\end{split}
\end{align}

\section*{Problems}

\subsection*{1.1. A function that is $C^2$ but not $C^3$}

Let $g : \R \to \R$ be the function in Example 1.2(iii). Show that the function $h(x) = \int_0^x g(t)\ dt$ is $C^2$ but not $C^3$ at $x = 0$.

\textbf{Proof.} Recall from Example 1.2(iii) that $g'(x) = f(x)$ while $f'(0)$ is undefined. It is not difficult to check that $f(x)$ is continuously differentiable at $x = 0$ but $f'(x)$ is not defined at $x = 0$. $\square$

\section{Tangent vectors in $\R^n$ as derivations}

A vector in $\R^3$ at a point $p$ is usually represented as a $3\times1$ "column" matrix or as an arrow emanating from $p$. A secant plane to a surface in $\R^3$ is a plane determined by three points on the surface. Taking these three points to approach $p$ in the limit, we obtain the \emph{tangent plane} to the surface at $p$. This representation of a tangent plane unfortunately only works when the surface is embedded in Euclidean space. We would rather have a representation intrinsic to the surface (or manifold).

\subsection{The directional derivative}

We write elements of the tangent space $T_p\R^n$ as $\langle v^1, \ldots, v^n \rangle$. The line through a point $p = (p^1, \ldots, p^n)$ with direction $v = \langle v^1, \ldots, v^n \rangle$ in $\R^n$ has parametrization $c(t) = (p^1 + tv^1, \ldots, p^n + tv^n)$. If $f$ is $C^\infty$ in a neighborhood of $p$ in $\R^n$ and $v \in T_p\R^n$, the \emph{directional derivative} of $f$ in the direction $v$ at $p$ is 

\begin{align}
    \displaystyle D_vf &= \lim_{t \to 0} \frac{f(c(t)) - f(p)}{t} = \left.\frac{d}{dt}\right|_{t=0} f(c(t)) = \sum_{i=1}^n v^i \frac{\partial f}{\partial x^i}(p).
\end{align}

We may unambiguously write

\begin{align}
    D_v = \sum v^i \left.\frac{\partial}{\partial x^i}\right|_p.
\end{align}

It's clear that $D : v \mapsto D_v$ creates a correspondence between tangent vectors and differential operators on functions so that we can study the latter.

\subsection{Germs of functions}

\end{document}