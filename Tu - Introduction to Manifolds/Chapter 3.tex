\documentclass[a4paper]{article}

\usepackage[margin=1in]{geometry}
\usepackage{amsfonts, amsmath, mathrsfs}
\usepackage{hyperref}

\newcommand{\C}{\mathbb{C}}
\newcommand{\R}{\mathbb{R}}
\newcommand{\Q}{\mathbb{Q}}
\newcommand{\Z}{\mathbb{Z}}

\begin{document}

\begin{center}
\LARGE{Loring Tu - Introduction to Manifolds}

\Large{Chapter 3: The tangent space}

\large{Carter Hinsley's notes}

Rendered \today
\end{center}

\setcounter{section}{7}
\section{The Tangent Space}

There are two equivalent ways to define a tangent vector at a point $p$ in an open set $U$ in $\R^n$:

\begin{itemize}
    \item An arrow represented by a column vector.
    \item A point-derivation of $C_p^\infty$, the algebra of germs of smooth functions at $p$.
\end{itemize}

\noindent The second of these ways is what we shall use, though both generalize to tangent vectors on manifolds.

\end{document}