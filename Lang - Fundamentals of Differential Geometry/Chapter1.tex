\documentclass[a4paper]{article}

\usepackage[margin=1in]{geometry}
\usepackage{amsfonts, amsmath}

\newcommand{\C}{\mathbb{C}}
\newcommand{\R}{\mathbb{R}}
\newcommand{\Q}{\mathbb{Q}}
\newcommand{\Z}{\mathbb{Z}}

\begin{document}

\begin{center}
\Large{Serge Lang - Fundamentals of Differential Geometry}

\large{Carter Hinsley's notes}

Rendered \today
\end{center}

\part{General differential theory}

\section{Differential calculus}

\subsection{Categories}

Most of this exposition is standard fare; the only term I have not seen before is the following.

\textbf{Definition.} Let $f : X \to Y$ be a morphism. A \emph{section} of $f$ is a morphism $g : Y \to X$ such that $f \circ g = \text{id}_Y$.

\subsection{Topological vector spaces}

A topological vector space $E$ over $\R$ is \textbf{a vector space} with a topology \textbf{where addition and scalar multiplication are continuous}. We assume such spaces are Hausdorff and locally convex (local in the sense that every nbhd of $0$ contains an open convex nbhd $U$ of $0$).

The category TVS has topological vector spaces as its objects and continuous linear maps as its morphisms. Lang denotes the hom-sets $L(\mathbf{E}, \mathbf{F})$. The continuous $r$-multilinear maps $\psi: \mathbf{E} \times \ldots \times \mathbf{E} \to \mathbf{F}$ are denoted $L^r(\mathbf{E}, \mathbf{F})$. The symmetric and alternating $r$-multilinear maps are denoted $L_s^r(\mathbf{E}, \mathbf{F})$ and $L_{\text{sym}}^r(\mathbf{E}, \mathbf{F})$ respectively. Isomorphisms in TVS are called toplinear isomorphisms. $\text{Lis}(\mathbf{E}, \mathbf{F})$ denotes the toplinear isomorphisms of $\mathbf{E}$ onto $\mathbf{F}$, while $\text{Laut}(\mathbf{E})$ denotes the toplinear automorphisms of $\mathbf{E}$. If a hom-set is written without a target object, the target is assumed to be $\R$. These are called \emph{forms} (particularly, continuous forms, $r$-multilinear forms, etc.).

\textbf{Definition (Cauchy net in a metric space).} A net $(x_\alpha)_{\alpha \in J}$ is \emph{Cauchy} if for any $\varepsilon > 0$ there exists $\alpha \in J$ such that $\alpha \preceq \beta, \gamma \implies d(x_\beta, x_\gamma) < \varepsilon$.

\textbf{Definition (complete space).} \emph{Complete spaces} are those spaces in which Cauchy nets (equivalently, Cauchy filters) converge. Note that we have not defined Cauchy nets outside of the context of metric spaces (the reason for this is that we need a notion of Cauchy filters to make this definition, which I haven't read about yet (TODO)). Also, we need only Cauchy sequences (as Cauchy nets) to converge in order for a metric space to be complete; it is not necessary to consider nets in general.

\textbf{Banach(able) spaces.} A complete TVS whose topology can be defined by a norm is called a Banachable space. When equipped with a norm, this is called a Banach space.

\textbf{Remark.} Because norms induce metrics, it is sufficient to think of the completeness property of Banach spaces in terms of convergence of Cauchy sequences; the machinery of Cauchy nets and Cauchy filters is extraneous here.

\subsection{Derivatives and composition of maps}

\subsection{Integration and Taylor's formula}

\subsection{The inverse mapping theorem}

\section{Manifolds}

\subsection{Atlases, charts, morphisms}

\subsection{Submanifolds, immersions, submersions}

\subsection{Partitions of unity}

\subsection{Manifolds with boundary}

\section{Vector bundles}

\subsection{Definition, pull backs}

\subsection{The tangent bundle}

\subsection{Exact sequences of bundles}

\subsection{Operations on vector bundles}

\subsection{Splitting of vector bundles}

\section{Vector fields and differential equations}

\subsection{Existence theorem for differential equations}

\subsection{Vector fields, curves, and flows}

\subsection{Sprays}

\subsection{The flow of a spray and the exponential map}

\subsection{Existence of tubular neighborhoods}

\subsection{Uniqueness of tubular neighborhoods}

\section{Operations on vector fields and differential forms}

\subsection{Vector fields, differential operators, brackets}

\subsection{Lie derivative}

\subsection{Exterior derivative}

\subsection{The Poincar\'e lemma}

\subsection{Contractions and Lie derivative}

\subsection{Vector fields and 1-forms under self duality}

\subsection{The canonical 2-form}

\subsection{Darboux's theorem}

\section{The theorem of Frobenius}

\subsection{Statement of the theorem}

\subsection{Differential equations depending on a parameter}

\subsection{Proof of the theorem}

\subsection{The global formulation}

\subsection{Lie groups and subgroups}

\part{Metrics, covariant derivatives, and Riemannian geometry}

\section{Metrics}

\subsection{Definition and functoriality}

\subsection{The Hilbert group}

\subsection{Reduction to the Hilbert group}

\subsection{Hilbertian tubular neighborhoods}

\subsection{The Morse-Palais lemma}

\subsection{The Riemannian distance}

\subsection{The canonical spray}

\section{Covariant derivatives and geodesics}

\subsection{Basic properties}

\subsection{Sprays and covariant derivatives}

\subsection{Derivative along a curve and parallelism}

\subsection{The metric derivative}

\subsection{More local results on the exponential map}

\subsection{Riemannian geodesic length and completeness}

\section{Curvature}

\subsection{The Riemann tensor}

\subsection{Jacobi lifts}

\subsection{Application of Jacobi lifts to $T\exp_x$}

\subsection{Convexity theorems}

\subsection{Taylor expansions}

\section{Jacobi lifts and tensorial splitting of the double tangent bundle}

\subsection{Convexity of Jacobi lifts}

\subsection{Global tubular neighborhood of a totally geodesic submanifold}

\subsection{More convexity and comparison results}

\subsection{Splitting of the double tangent bundle}

\subsection{Tensorial derivative of a curve in $TX$ and of the exponential map}

\subsection{The flow and the tensorial derivative}

\section{Curvature and the variation formula}

\subsection{The index form, variations, and the second variation formula}

\subsection{Growth of a Jacobi lift}

\subsection{The semi parallelogram law and negative curvature}

\subsection{Totally geodesic submanifolds}

\subsection{Rauch comparison theorem}

\section{An example of seminegative curvature}

\subsection{$\text{Pos}_n(\R)$ as a Riemannian manifold}

\subsection{The metric increasing property of the exponential map}

\subsection{Totally geodesic and symmetric submanifolds}

\section{Automorphisms and symmetries}

\subsection{The tensorial second derivative}

\subsection{Alternative definitions of Killing fields}

\subsection{Metric Killing fileds}

\subsection{Lie algebra properties of Killing fields}

\subsection{Symmetric spaces}

\subsection{Parallelism and the Riemann tensor}

\section{Immersions and submersions}

\subsection{The covariant derivative on a submanifold}

\subsection{The Hessian and Laplacian on a submanifold}

\subsection{The covariant derivative on a Riemannian submersion}

\subsection{The Hessian and Laplacian on a Riemannian submersion}

\subsection{The Riemann tensor on submanifolds}

\subsection{The Riemann tensor on a Riemannian submersion}

\part{Volume forms and integration}

\section{Volume forms}

\subsection{Volume forms and the divergence}

\subsection{Covariant derivatives}

\subsection{The Jacobian determinant of the exponential map}

\subsection{The Hodge star on forms}

\subsection{Hodge decomposition of differential forms}

\subsection{Volume forms in a submersion}

\subsection{Volume forms on Lie groups and homogeneous spaces}

\subsection{Homogeneously fibered submersions}

\subsection*{Appendix. Direct image of differential operators}

\section{Integration of differential forms}

\subsection{Sets of measure $0$}

\subsection{Change of variables formula}

\subsection{Orientation}

\subsection{The measure associated with a differential form}

\subsection{Homogeneous spaces}

\section{Stokes' theorem}

\subsection{Stokes' theorem for a rectangular simplex}

\subsection{Stokes' theorem on a manifold}

\subsection{Stokes' theorem with singularities}

\section{Applications of Stokes' theorem}

\subsection{The maximal de Rham cohomology}

\subsection{Moser's theorem}

\subsection{The divergence theorem}

\subsection{The adjoint of $d$ for higher degree forms}

\subsection{Cauchy's theorem}

\subsection{The residue theorem}

\part*{Appendix. The spectral theorem}

\section{Hilbert space}

\section{Functionals and operators}

\section{Hermitian operators}

\end{document}
