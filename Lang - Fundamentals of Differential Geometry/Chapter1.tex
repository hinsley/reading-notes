\documentclass[a4paper]{article}

\usepackage[margin=1in]{geometry}
\usepackage{amsfonts, amsmath}

\newcommand{\C}{\mathbb{C}}
\newcommand{\R}{\mathbb{R}}
\newcommand{\Q}{\mathbb{Q}}
\newcommand{\Z}{\mathbb{Z}}

\begin{document}

\begin{center}
\Large{Serge Lang - Fundamentals of Differential Geometry}

\large{Carter Hinsley's notes}
\end{center}

\part{General differential theory}

\section{Differential calculus}

\subsection{Categories}

\subsection{Topological vector spaces}

A topological vector space $E$ over $\R$ is a vector space with a topology where addition and scalar multiplication are continuous. We assume such spaces are Hausdorff and locally convex (local in the sense that every nbhd of $0$ contains an open convex nbhd $U$ of $0$).

The category TVS has topological vector spaces as its objects and continuous linear maps as its morphisms. Lang denotes the hom-sets $L(\mathbf{E}, \mathbf{F})$. The continuous $r$-multilinear maps $\psi: \mathbf{E} \times \ldots \times \mathbf{E} \to \mathbf{F}$ are denoted $L^r(\mathbf{E}, \mathbf{F})$. The symmetric and alternating $r$-multilinear maps are denoted $L_s^r(\mathbf{E}, \mathbf{F})$ and $L_{\text{sym}}^r(\mathbf{E}, \mathbf{F})$ respectively. Isomorphisms in TVS are called toplinear isomorphisms. $\text{Lis}(\mathbf{E}, \mathbf{F})$ denotes the toplinear isomorphisms of $\mathbf{E}$ onto $\mathbf{F}$, while $\text{Laut}(\mathbf{E})$ denotes the toplinear automorphisms of $\mathbf{E}$. If a hom-set is written without a target object, the target is assumed to be $\R$. These are called \emph{forms} (particularly, continuous forms, $r$-multilinear forms, etc.).

\emph{Complete spaces.} 

\emph{Banach(able) spaces.} Any complete TVS whose topology can be defined by a norm is called a Banachable space. When equipped with the norm, this is called a Banach space.

\subsection{Derivatives and composition of maps}

\subsection{Integration and Taylor's formula}

\subsection{The inverse mapping theorem}

\section{Manifolds}

\subsection{Atlases, charts, morphisms}

\subsection{Submanifolds, immersions, submersions}

\subsection{Partitions of unity}

\subsection{Manifolds with boundary}

\section{Vector bundles}

\subsection{Definition, pull backs}

\subsection{The tangent bundle}

\subsection{Exact sequences of bundles}

\subsection{Operations on vector bundles}

\subsection{Splitting of vector bundles}

\section{Vector fields and differential equations}

\subsection{Existence theorem for differential equations}

\subsection{Vector fields, curves, and flows}

\subsection{Sprays}

\subsection{The flow of a spray and the exponential map}

\subsection{Existence of tubular neighborhoods}

\subsection{Uniqueness of tubular neighborhoods}

\section{Operations on vector fields and differential forms}

\subsection{Vector fields, differential operators, brackets}

\subsection{Lie derivative}

\subsection{Exterior derivative}

\subsection{The Poincar\'e lemma}

\subsection{Contractions and Lie derivative}

\subsection{Vector fields and 1-forms under self duality}

\subsection{The canonical 2-form}

\subsection{Darboux's theorem}

\section{The theorem of Frobenius}

\subsection{Statement of the theorem}

\subsection{Differential equations depending on a parameter}

\subsection{Proof of the theorem}

\subsection{The global formulation}

\subsection{Lie groups and subgroups}

\end{document}
