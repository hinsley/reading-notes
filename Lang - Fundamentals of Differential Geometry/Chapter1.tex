\documentclass[a4paper]{article}

\usepackage[margin=1in]{geometry}
\usepackage{amsfonts}

\newcommand{\C}{\mathbb{C}}
\newcommand{\R}{\mathbb{R}}
\newcommand{\Q}{\mathbb{Q}}
\newcommand{\Z}{\mathbb{Z}}

\begin{document}

\begin{center}
\Large{Serge Lang - Fundamentals of Differential Geometry}

\large{Carter Hinsley's notes}
\end{center}

\pagebreak

\part{General differential theory}

\section{Differential calculus}

\subsection{Categories}

\subsection{Topological vector spaces}

A topological vector space $E$ over $\R$ is a vector space with a topology where addition and scalar multiplication are continuous. We assume such spaces are Hausdorff and locally convex (local in the sense that every nbhd of $0$ contains an open nbhd $U$ of $0$ wherein for every point $x, y \in U$ and $0 \leq t \leq 1$, $tx+(1-t)y$ also lies in $U$).

\subsection{Derivatives and composition of maps}

\subsection{Integration and Taylor's formula}

\subsection{The inverse mapping theorem}

\section{Manifolds}

\subsection{Atlases, charts, morphisms}

\subsection{Submanifolds, immersions, submersions}

\subsection{Partitions of unity}

\subsection{Manifolds with boundary}

\section{Vector bundles}

\subsection{Definition, pull backs}

\subsection{The tangent bundle}

\subsection{Exact sequences of bundles}

\subsection{Operations on vector bundles}

\subsection{Splitting of vector bundles}

\section{Vector fields and differential equations}

\subsection{Existence theorem for differential equations}

\subsection{Vector fields, curves, and flows}

\subsection{Sprays}

\subsection{The flow of a spray and the exponential map}

\subsection{Existence of tubular neighborhoods}

\subsection{Uniqueness of tubular neighborhoods}

\section{Operations on vector fields and differential forms}

\subsection{Vector fields, differential operators, brackets}

\subsection{Lie derivative}

\subsection{Exterior derivative}

\subsection{The Poincar\'e lemma}

\subsection{Contractions and Lie derivative}

\subsection{Vector fields and 1-forms under self duality}

\subsection{The canonical 2-form}

\subsection{Darboux's theorem}

\section{The theorem of Frobenius}

\subsection{Statement of the theorem}

\subsection{Differential equations depending on a parameter}

\subsection{Proof of the theorem}

\subsection{The global formulation}

\subsection{Lie groups and subgroups}

\end{document}
