\documentclass[a4paper]{article}

\usepackage[margin=1in]{geometry}
\usepackage{amsfonts, amsmath, mathrsfs}
\usepackage{indentfirst}
\usepackage{hyperref}

\newcommand{\C}{\mathbb{C}}
\newcommand{\R}{\mathbb{R}}
\newcommand{\Q}{\mathbb{Q}}
\newcommand{\Z}{\mathbb{Z}}

\begin{document}

\begin{center}
\LARGE{Sheldon Axler - Measure, Integration \& Real Analysis}

\Large{Chapter 2: Measures}

\large{Carter Hinsley's notes}

Rendered \today
\end{center}

\section{Outer measure on $\R$}

We have defined the outer measure $|A|$ of a set $A \subseteq \R$ as
\[|A| = \inf\left\{\sum_{k=1}^\infty m(I_k) \mid I_1, I_2, \ldots \text{ are open intervals such that } A \subseteq \bigcup_{k=1}^\infty I_k\right\}.\]
The first novel result in this chapter states that the outer measure is not additive:

\textbf{Theorem 2.18.} There exist disjoint subsets $A, B \subset \R$ such that $|A \cup B| \neq |A| + |B|$.

\emph{Proof.} If $a \in [-1, 1]$, let $\tilde{a} = \{c \in [-1, 1] \mid a - c \in \Q\}$. We have $[-1, 1] = \bigcup_{a \in [-1, 1]} \tilde{a}$. Invoking the axiom of choice, let $V$ be a set containing a representative element of $\tilde{a}$ for each distinct $a \in [-1, 1]$. Enumerate the elements of $[-2, 2] \cap \Q$ as $\{r_1, r_2, \ldots\}$.

Axler claims that it then follows that $[-1, 1] \subset \bigcup_{k=1}^\infty (r_k + V)$. But why? This is because, for any $v \in V$, we have $\tilde{v} \subset \bigcup_{k=1}^\infty (r_k + v)$. Since $\tilde{v} = \tilde{a}$ for a unique $a \in [-1, 1]$ and each such $a \in [-1, 1]$ has a corresponding $v \in V$, this gives the desired result.

We then have by countable subadditivity, the order preservation of outer measure, and translation invariance the inequality $2 = |[-1, 1]| \leq \sum_{k=1}^\infty |r_k + V| = \sum_{k=1}^\infty |V| = \infty$. Hence $|V| > 0$.

The sets $r_1 + V, r_2 + V, \ldots$ are disjoint (recall the construction of $V$). Let $n \in \Z^+$. Then $\bigcup_{k=1}^n (r_k + V) \subset [-3, 3]$. This implies that $\left|\bigcup_{k=1}^n (r_k + V)\right| \leq 6$. Note that $\sum_{k=1}^n |r_k + V| = \sum_{k=1}^n |V| = n|V|$. By Archimedean principle we may choose $n$ so that $n|V| > 6$. It follows then that $\left|\bigcup_{k=1}^n (r_k + V)\right| < \sum_{k=1}^n |r_k + V|$. Because these $r_k + V$ are disjoint, we observe by induction that we have proven the theorem. $\square$

The result above uses a \emph{Vitali set}.

I have read in some competing notes for this material that the outer measure is sometimes defined using closed intervals rather than open intervals. This allows one to speak about coverings of sets being \emph{almost disjoint}, which could prove helpful.

\section*{Exercises}

\textbf{1} Prove that if $A$ and $B$ are subsets of $\R$ and $|B| = 0$, then $|A \cup B| = |A|$.

\emph{Proof.} Observe that $A \subseteq A \cup B$ so that $|A| \leq |A \cup B|$. Meanwhile, subadditivity yields $|A \cup B| \leq |A| + |B| = |A|$ so that $|A| = |A \cup B|$. $\square$

\textbf{2} Suppose $A \subset \R$ and $t \in \R$. Let $tA = \{ta \mid a \in A\}$. Prove that $|tA| = |t||A|$. [Assume that $0 \cdot \infty$ is defined to be $0$.]

\emph{Proof.} The case wherein $t = 0$ is trivial; $tA = \{0\}$ or is an empty set. The case wherein $t < 0$ is virtually identical to when $t > 0$, so we omit it. Let $\{I_k\}$ be a countable cover of $A$ by open intervals. For any $a \in A$, we have $a \in I_k$ for at least one $k \in \Z^+$. It follows that $ta \in tI_k$ so that $tA \subseteq \bigcup_{k=1}^\infty tI_k$. Likewise, if $tA$ is covered by $\{tI_k\}$ then $A$ is covered by $\{I_k\}$. Thus $|tA| = \inf\left\{\sum_{k=1}^\infty tI_k \mid I_1, I_2, \ldots \text{ are open intervals and } A \subseteq \bigcup_{k=1}^\infty I_k\right\} = t|A| = |t||A|$. $\square$

\textbf{3} Prove that if $A, B \subset \R$ and $|A| < \infty$, then $|B - A| \geq |B| - |A|$.

\emph{Proof.} Observe that $B = (B - A) \cup (A \cap B)$. By subadditivity and order preservation we have $|B| - |A| = |(B - A) \cup (A \cap B)| - |A| \leq |B - A| + |A \cap B| - |A| = |B - A| - (|A| - |A \cap B|) \leq |B - A|$. $\square$

\textbf{4} Suppose $F$ is a subset of $\R$ with the property that every open cover of $F$ has a finite subcover. Prove that $F$ is closed and bounded.

\emph{Proof.} Obviously a set must be bounded in order to be compact. We now show that any non-closed set has an open set with no finite subcover. If a set $E$ is not closed, there is a limit point $c \in \R$ of $E$ such that $c \notin E$; particularly, given any radius $r > 0$, there exists a point $x \in E$ such that $0 < |x - c| < r$. Let $\{x_k\}$ be a sequence of points in $E$ such that $|x_{k+1} - c| \leq |x_k - c|$ for all $k \in \Z^+$ and $x_k \to x$ as $k \to \infty$. Let $\{I_k\}$ be a countable covering of $E$ by open intervals such that $x_i \in I_j$ if and only if $i = j$ and $x \notin \bigcup_{k=1}^\infty I_k$ (these criteria are possible because $\R$ is Hausdorff). $\{I_k\}$ is a ``minimal'' covering of $E$ in the sense that if any $I_k$ were removed, the sequence would no longer form a cover of $E$. As $\{I_k\}$ is an uncountable open cover, $E$ is not compact. $\square$

\textbf{5} Suppose $\mathcal{A}$ is a set of closed subsets of $\R$ such that $\bigcap_{F \in \mathcal{A}} F = \emptyset$. Prove that if $\mathcal{A}$ contains at least one bounded set, then there exist $n \in \Z^+$ and $F_1, \ldots, F_n \in \mathcal{A}$ such that $F_1 \cap \ldots \cap F_n = \emptyset$.

\emph{Proof.} Suppose $\mathcal{A}$ contains a bounded set $S$. Then
\begin{align}
\begin{split}
    S \cap \bigcap_{F \in \mathcal{A}} F &= \emptyset \\
    \implies S \subseteq \overline{\bigcap_{F \in \mathcal{A}} F} &= \bigcup_{F \in \mathcal{A}} \overline{F}.
\end{split}
\end{align}
By Heine-Borel theorem, $S$ is compact. Using this result, and because $\{\overline{F}\}_{F \in \mathcal{A}}$ is an open cover of $S$, there exists a finite subcover $\{\overline{F_k}\}_{k=2}^n$ of $S$. Hence
\begin{align}
\begin{split}
    S \subseteq \bigcup_{k=2}^n \overline{F_k} &= \overline{\bigcap_{k=2}^n F_k} \\
    \implies S \cap \bigcap_{k=2}^n F_k &= \emptyset.
\end{split}
\end{align}
Letting $F_1 = S$, we have attained the desired result. $\square$

\textbf{6} Prove that if $a, b \in \R$ and $a < b$, then $|(a, b)| = |[a, b)| = |(a, b]| = b - a$.

\emph{Proof.} Let $\varepsilon > 0$ satisfy $2\varepsilon \leq b - a$ but otherwise be chosen arbitrarily. Note that by the order preservation of outer measure we have $b - a - 2\varepsilon = |[a + \varepsilon, b - \varepsilon]| \leq |I| \leq |[a, b]| = b - a$ for $I = (a, b), I = [a, b),$ and $I = (a, b]$. Concisely, we have $b - a - 2\varepsilon \leq |I| \leq b - a$. Since $\varepsilon$ may be arbitrarily small, we may take the limit: $\displaystyle \lim_{\varepsilon \to 0^+} b - a - 2\varepsilon \leq \lim_{\varepsilon \to 0^+} |I| \leq \lim_{\varepsilon \to 0^+} b - a$. Thus $b - a \leq |I| \leq b - a$ so that $|I| = b - a$. $\square$

\textbf{7} Suppose $a, b, c, d$ are real numbers with $a < b$ and $c < d$. Prove that $|(a, b) \cup (c, d)| = (b - a) + (d - c)$ if and only if $(a, b) \cap (c, d) = \emptyset$.

\emph{Proof.} By subadditivity we know that $|(a, b) \cup (c, d)| \leq (b - a) + (d - c)$.

$(\impliedby)$ Suppose that $(a, b) \cap (c, d) = \emptyset$. Then
\begin{align}
\begin{split}
    |(a, b) \cap (c, d)| &= \inf\left\{\left.\sum_{k=1}^\infty |I_k| \right| I_1, I_2, \ldots \text{ are open intervals and } (a, b) \cap (c, d) \subseteq \bigcup_{k=1}^\infty I_k\right\} \\ \ \\
    &\geq \inf\left\{\left.\sum_{k=1}^\infty |I_k| \right| \ldots \text{and } (a, b) \subseteq \bigcup_{k=1}^\infty I_k\right\} + \inf\left\{\left.\sum_{k=1}^\infty +I_k| \right| \ldots \text{and } (c, d) \subseteq \bigcup_{k=1}^\infty I_k\right\} \\ \ \\
    &= |(a, b)| + |(c, d)| = (b - a) + (d - c).
\end{split}
\end{align}

$(\implies)$ On the other hand, suppose that $(a, b) \cap (c, d) = (c, b) \neq \emptyset$. Then $(a, b) \cup (c, d) = (a, d)$ so that $|(a, b) \cup (c, d)| = d - a < (d - a) + (b - c) = (b - a) + (d - c)$. $\square$

\textbf{8} Prove that if $A \subset \R$ and $t > 0$, then $|A| = |A \cap (-t, t)| + |A \cap (\R - (-t, t))|$.

\emph{Proof.} In general, subdivision of a set by intersection with finitely many disjoint intervals will satisfy additivity of outer measure. To show this for the problem at hand, write
\begin{align}
\begin{split}
    |A| &= \inf\left\{\left.\sum_{k=1}^\infty |I_k| \right| I_1, I_2, \ldots \text{ are open intervals and } A \subseteq \bigcup_{k=1}^\infty I_k\right\} \\ \ \\
    &= \inf\left\{\left.\sum_{k=1}^\infty (|I_k \cap (-\infty, -t)| + |I_k \cap (-t, t)| + |I_k \cap (t, \infty)|) \right| \ldots \text{and } A \subseteq \bigcup_{k=1}^\infty I_k\right\}.
\end{split}
\end{align}
To see that additivity holds here, observe that if any interval $I_k$ has $t$ (or $-t$) in its interior, the outer measure of that interval is the same as the sum of outer measures $I_k \cap (-\infty, t)$ and $I_k \cap (t, \infty)$ (or the sum of $I_k \cap (-\infty, -t)$ and $I_k \cap (-t, \infty)$). Thus
\begin{align}
\begin{split}
    |A| &= \inf\left\{\left.\sum_{k=1}^\infty |I_k \cap (-\infty, -t)| \right| \ldots \text{and } A \subseteq \bigcup_{k=1}^\infty I_k\right\} \\ \ \\
    &+ \inf\left\{\left.\sum_{k=1}^\infty |I_k \cap (-t, t)| \right| \ldots \text{and } A \subseteq \bigcup_{k=1}^\infty I_k\right\} \\ \ \\
    &+ \inf\left\{\left.\sum_{k=1}^\infty |I_k \cap (t, \infty)| \right| \ldots \text{and } A \subseteq \bigcup_{k=1}^\infty I_k\right\} \\ \ \\
    &= |A \cap (-t, t)| + |A \cap (-\infty, -t)| + |A \cap (t, \infty)|.
\end{split}
\end{align}
By the same logic, we have
\begin{align}
\begin{split}
    |A \cap (-t, t)| + |A \cap \overline{(-t, t)}| &= |A \cap (-t, t)| + |A \cap ((-\infty, -t) \cup (t, \infty))| \\ \ \\
    &= |A \cap (-t, t)| + |A \cap (-\infty, -t)| + |A \cap (t, \infty)|.
\end{split}
\end{align}
Hence $|A| = |A \cap (-t, t)| + |A \cap \overline{(-t, t)}|$. $\square$

\textbf{9} Prove that $|A| = \lim_{t \to \infty} |A \cap (-t, t)|$ for all $A \subset \R$.

\emph{Proof.} Suppose that $|A| < \infty$. Then, for any $\varepsilon > 0$, there exists a cover of $A$ by open intervals such that $\displaystyle \sum_{k=1}^\infty |I_k| = |A| + \varepsilon$. That is, $\displaystyle \lim_{n \to \infty} \sum_{k=1}^n |I_k| = |A| + \varepsilon$. Looking a little more carefully, for any $\varepsilon' > 0$, there exists $M \in \Z^+$ such that $n \geq M$ implies $\displaystyle \left(\sum_{k=1}^n |I_k|\right) - |A| - \varepsilon < \varepsilon'$, or that $\displaystyle \left(\sum_{k=1}^n |I_k|\right) - |A| < \varepsilon + \varepsilon'$. Essentially, no matter what ordering is chosen on the sequence of intervals, we can find a finite subsequence that has total measure arbitrarily close to $|A|$. Of course, the corresponding union $\bigcup_{k=1}^n I_k$ is a bounded set. Fixing $\varepsilon + \varepsilon'$, we can always find a bound $t$ so that $\bigcup_{k=1}^n I_k \subseteq (-t, t)$, which yields the desired result. $\square$

\textbf{10} Prove that $|[0, 1] - \Q| = 1$.

\emph{Proof.} Because $[0, 1] - \Q \subset (0, 1)$, we have $|[0, 1] - \Q| \leq |(0, 1)|$. Because any open cover of $[0, 1] - \Q$ is also an open cover of $(0, 1)$, the fact that outer measure is an infimum yields $|[0, 1] - \Q| \geq |(0, 1)|$. Hence $|[0, 1] - \Q| = |(0, 1)| = 1$. $\square$

\textbf{11} Prove that if $I_1, I_2, \ldots$ is a disjoint sequence of open intervals, then $\left|\bigcup_{k=1}^\infty I_k\right| = \sum_{k=1}^\infty m(I_k)$

\emph{Proof.} For any $n \in \Z^+$, we have $\displaystyle \left|\bigcup_{k=1}^n I_k\right| \leq \left|\bigcup_{k=1}^\infty I_k\right|$ by order preservation of outer measure. Also, for any $n \in \Z^+$ we have $\sum_{k=1}^n |I_k| = \left|\bigcup_{k=1}^n I_k\right|$. Taking limits and using countable subadditivity, we get $\displaystyle \sum_{k=1}^\infty |I_k| = \lim_{n \to \infty} \left|\bigcup_{k=1}^n I_k\right| \leq \left|\bigcup_{k=1}^\infty I_k\right| \leq \sum_{k=1}^\infty |I_k|$. Hence $\displaystyle \left|\bigcup_{k=1}^\infty I_k\right| = \sum_{k=1}^\infty |I_k|$. $\square$

\textbf{12} Suppose $r_1, r_2, \ldots$ is a sequence that contains every rational number. Let $F = \R - \bigcup_{k=1}^\infty \left(r_k - \frac{1}{2^k}, r_k + \frac{1}{2^k}\right)$.

(a) Show that $F$ is a closed subset of $\R$.

\emph{Proof.} This is a basic topological fact; open sets are closed under arbitrary (here countable) union. $\square$

(b) Prove that if $I$ is an interval contained in $F$, then $I$ contains at most one element.

\emph{Proof.} The rationals are dense in $\R$; every interval with more than one element contains a rational. Because no rationals are in $F$, the only intervals must be singleton sets. $\square$

(c) Prove that $|F| = \infty$.

\emph{Proof.} Firstly, we show that $|\R| = \infty$. Let $I_k = \left(k - \frac12, k + \frac12\right)$ for all $k \in \Z^+$. By exercise 11 and order preservation we have $\infty = \sum_{k=1}^\infty 1 = \left|\bigcup_{k=1}^\infty I_k\right| \leq |\R|$; this yields the desired result. Note now that $F \cup \overline{F} = \R$. By finite subadditivity we have $\infty = |\R| = |F \cup \overline{F}| \leq |F| + |\overline{F}|$. Thus $\infty = |F| + |\overline{F}|$.

It now suffices to show that $|\overline{F}| < \infty$. Note that $\overline{F} = \bigcup_{k=1}^\infty \left(r_k - \frac{1}{2^k}, r_k + \frac{1}{2^k}\right)$. By countable subadditivity we have $|\overline{F}| \leq \sum_{k=1}^\infty \frac{1}{2^{k-1}} = 2$. Hence $|F| = \infty$. $\square$

\textbf{13} Suppose $\varepsilon > 0$. Prove that there exists a subset $F$ of $[0, 1]$ such that $F$ is closed, every element of $F$ is an irrational number, and $|F| > 1 - \varepsilon$.

\emph{Proof.} Because $|\Q| = 0$, we can fix $\varepsilon > 0$ and find an open interval covering sequence $\{I_k\}$ of $[0, 1] \cap \Q$ such that $\sum_{k=1}^\infty |I_k| < \varepsilon$. Let $F = [0, 1] - \bigcup_{k=1}^\infty I_k$. Then $F$ is closed and contains only irrationals. Note that $[0, 1] = F \cup \bigcup_{k=1}^\infty I_k$. By countable subadditivity, we obtain $1 = |[0, 1]| = \left|F \cup \bigcup_{k=1}^\infty I_k\right| \leq |F| + \sum_{k=1}^\infty |I_k| < |F| + \varepsilon$. That is, $|F| > 1 - \varepsilon$. $\square$

\section{Measurable spaces and functions}

There is no way to get a notion of measure which has all of the nice properties of outer measure (non-negativity, monotonicity, agreement with Jordan measure on open intervals, and translation invariance) as well as additivity on countable disjoint unions. This is because the proof we used for outer measure -- discussing the same result there -- works with just these properties.

\subsection{$\sigma$-algebras}

\textbf{Definition.} Let $X$ be a set and $S \subset \mathcal{P}(X)$. $S$ is called a \emph{$\sigma$-algebra} on $X$ if $\emptyset \in S$ and $S$ is closed with respect to complements and countable unions.

\textbf{Example.} Let $X$ be a set. The collection $S$ of all subsets $E$ of $X$ such that $E$ is countable or $X - E$ is countable is a $\sigma$-algebra on $X$.

\emph{Proof.} $S$ obviously contains $\emptyset$ and is closed under complementation. We have only to check that $S$ is closed under countable union. Let $E = \bigcup_{k=1}^\infty E_k$ where each $E_k$ is a member of $S$. If each $E_k$ is countable, then as the countable union of countable sets is countable, $E$ is countable and thus $E \in S$. Suppose on the other hand WLOG that $E_1$ is uncountable; it therefore must have countable complement. Then $\overline{E} = \overline{\bigcup_{k=1}^\infty E_k} = \bigcap_{k=1}^\infty \overline{E_k} \subseteq \overline{E_1}$. As $\overline{E_1}$ is countable, so too must be $\overline{E}$. Hence $E \in S$. $\square$

\textbf{Remark.} $\sigma$-algebras are closed with respect to countable intersection and set subtraction.

\textbf{Definition.} A \emph{measurable space} is an ordered pair $(X, S)$, where $X$ is a set and $S$ is a $\sigma$-algebra on $X$. Any element of $S$ is called an \emph{$S$-measurable set}.

\textbf{Theorem 2.27.} Let $X$ be a set and $\mathcal{A} \subseteq \mathcal{P}(X)$. The intersection of all $\sigma$-algebras on $X$ that contain $\mathcal{A}$ is a $\sigma$-algebra on $X$. The proof of this statement is very simple.

\textbf{Definition.} The smallest $\sigma$-algebra on $\R$ containing the standard topology on $\R$ is called the \emph{Borel $\sigma$-algebra} on $\R$. Elements of this $\sigma$-algebra are called \emph{Borel sets}.

\textbf{Example.} This was an example of a Borel set one of my professors showed me a while back but I did not have the background to fully appreciate. Any half-open interval $[a, b)$ in $\R$ is a Borel set as $[a, b) = \bigcap_{k=1}^\infty \left(a - \frac1k, b\right)$.

\textbf{Example.} If $f : \R \to \R$ is a function, the set of points at which $f$ is continuous is a Borel set (it is the intersection of a sequence of open sets; see Exercise 12 in this section).

\subsection{Inverse images}

\textbf{Proposition 2.33 (algebra of inverse images).} Suppose $f : X \to Y$ is a function. Then

(a) $f^{-1}(Y - A) = X - f^{-1}(A)$ for every $A \subseteq Y$ (inverse images preserve set subtraction);

(b) $f^{-1}\left(\bigcup_{A \in \mathcal{A}} f^{-1}(A)\right) = \bigcup_{A \in \mathcal{A}} f^{-1}(A)$ for every $\mathcal{A} \subseteq \mathcal{P}(Y)$ (inverse images preserve arbitrary unions);

(c) $f^{-1}\left(\bigcap_{A \in \mathcal{A}} f^{-1}(A)\right) = \bigcap_{A \in \mathcal{A}} f^{-1}(A)$ for every $\mathcal{A} \subseteq \mathcal{P}(Y)$ (inverse images preserve arbitrary intersections).

\textbf{Proposition 2.34.} Suppose $f : X \to Y$ and $g : Y \to W$ are functions. Then $(g \circ f)^{-1}(A) = f^{-1}(g^{-1}(A))$ for every $A \subseteq W$.

\subsection{Measurable functions}

\textbf{Definition.} Suppose $(X, S)$ is a measurable space. A function $f : X \to \R$ is called \emph{$S$-measurable} if $f^{-1}(B) \in S$ for every Borel set $B \subseteq \R$.

Measurability resembles continuity in some fashion, which could make it easier to remember this definition.

\textbf{Example.} If $S = \{\emptyset, X\}$, the only $S$-measurable functions $X \to \R$ are the constant functions.

\textbf{Example.} If $S = \mathcal{P}(X)$, then all functions $X \to \R$ are $S$-measurable.

\textbf{Example.} If $S = \{\emptyset, (-\infty, 0), [0, \infty), \R\}$ (a $\sigma$-algebra on $\R$), then a function $f : \R \to \R$ is $S$-measurable if and only if $f$ is constant on $(-\infty, 0)$ and constant on $[0, \infty)$.

To get some idea of why this last example is true, let $f(x) = x$ be the identity function on $\R$. Then $f^{-1}(y) = y$ for all $y \in \R$. As $(-1, 1)$ is a Borel set in $\R$ (it's an open set, so it must be a Borel set), yet $f^{-1}((-1, 1)) = (-1, 1) \notin S$, we can see $f$ isn't $S$-measurable.

\textbf{Definition.} The indicator function for membership within a set $E$ is the \emph{characteristic function}, written $\chi_E$.

\textbf{Example.} $\chi_E$ is an $S$-measurable function if and only if $E \in S$.

In order to check if a function is $S$-measurable, the definition of $S$-measurability suggests we need to check if the inverse image of every Borel set in $\R$ maps back to an element of $S$. It turns out that we can actually just check a smaller collection of Borel sets in $\R$ for their inverse image, due to the following result.

\textbf{Theorem 2.39.} Suppose $(X, S)$ is a measurable space and $f : X \to \R$ is a function such that $f^{-1}((a, \infty)) \in S$ for all $a \in \R$. Then $f$ is a $S$-measurable function.

% TODO: Run through the proof of this.

\end{document}