\documentclass[a4paper]{article}

\usepackage[margin=1in]{geometry}
\usepackage{amsfonts, amsmath, mathrsfs}
\usepackage{indentfirst}
\usepackage{hyperref}

\newcommand{\C}{\mathbb{C}}
\newcommand{\R}{\mathbb{R}}
\newcommand{\Q}{\mathbb{Q}}
\newcommand{\Z}{\mathbb{Z}}

\begin{document}

\begin{center}
\LARGE{Sheldon Axler - Measure, Integration \& Real Analysis}

\Large{Chapter 2: Measures}

\large{Carter Hinsley's notes}

Rendered \today
\end{center}

\section{Outer measure on $\R$}

We have defined the outer measure $|A|$ of a set $A \subseteq \R$ as
\[|A| = \inf\left\{\sum_{k=1}^\infty m(I_k) \mid I_1, I_2, \ldots \text{ are open intervals such that } A \subseteq \bigcup_{k=1}^\infty I_k\right\}.\]
The first novel result in this chapter states that the outer measure is not additive:

\textbf{Theorem 2.18.} There exist disjoint subsets $A, B \subset \R$ such that $|A \cup B| \neq |A| + |B|$.

\emph{Proof.} If $a \in [-1, 1]$, let $\tilde{a} = \{c \in [-1, 1] \mid a - c \in \Q\}$. We have $[-1, 1] = \bigcup_{a \in [-1, 1]} \tilde{a}$. Invoking the axiom of choice, let $V$ be a set containing a representative element of $\tilde{a}$ for each distinct $a \in [-1, 1]$. Enumerate the elements of $[-2, 2] \cap \Q$ as $\{r_1, r_2, \ldots\}$.

Axler claims that it then follows that $[-1, 1] \subset \bigcup_{k=1}^\infty (r_k + V)$. But why? This is because, for any $v \in V$, we have $\tilde{v} \subset \bigcup_{k=1}^\infty (r_k + v)$. Since $\tilde{v} = \tilde{a}$ for a unique $a \in [-1, 1]$ and each such $a \in [-1, 1]$ has a corresponding $v \in V$, this gives the desired result.

We then have by countable subadditivity, the order preservation of outer measure, and translation invariance the inequality $2 = |[-1, 1]| \leq \sum_{k=1}^\infty |r_k + V| = \sum_{k=1}^\infty |V| = \infty$. Hence $|V| > 0$.

The sets $r_1 + V, r_2 + V, \ldots$ are disjoint (recall the construction of $V$). Let $n \in \Z^+$. Then $\bigcup_{k=1}^n (r_k + V) \subset [-3, 3]$. This implies that $\left|\bigcup_{k=1}^n (r_k + V)\right| \leq 6$. Note that $\sum_{k=1}^n |r_k + V| = \sum_{k=1}^n |V| = n|V|$. By Archimedean principle we may choose $n$ so that $n|V| > 6$. It follows then that $\left|\bigcup_{k=1}^n (r_k + V)\right| < \sum_{k=1}^n |r_k + V|$. Because these $r_k + V$ are disjoint, we observe by induction that we have proven the theorem. $\square$

I have read in some competing notes for this material that the outer measure is sometimes defined using closed intervals rather than open intervals. This allows one to speak about coverings of sets being \emph{almost disjoint}, which could prove helpful.

\section*{Exercises}

\textbf{1} Prove that if $A$ and $B$ are subsets of $\R$ and $|B| = 0$, then $|A \cup B| = |A|$.

\emph{Proof.} Observe that $A \subseteq A \cup B$ so that $|A| \leq |A \cup B|$. Meanwhile, subadditivity yields $|A \cup B| \leq |A| + |B| = |A|$ so that $|A| = |A \cup B|$. $\square$

\textbf{2} Suppose $A \subset \R$ and $t \in \R$. Let $tA = \{ta \mid a \in A\}$. Prove that $|tA| = |t||A|$. [Assume that $0 \cdot \infty$ is defined to be $0$.]

\emph{Proof.} The case wherein $t = 0$ is trivial; $tA = \{0\}$ or is an empty set. The case wherein $t < 0$ is virtually identical to when $t > 0$, so we omit it. Let $\{I_k\}$ be a countable cover of $A$ by open intervals. For any $a \in A$, we have $a \in I_k$ for at least one $k \in \Z^+$. It follows that $ta \in tI_k$ so that $tA \subseteq \bigcup_{k=1}^\infty tI_k$. Likewise, if $tA$ is covered by $\{tI_k\}$ then $A$ is covered by $\{I_k\}$. Thus $|tA| = \inf\left\{\sum_{k=1}^\infty tI_k \mid I_1, I_2, \ldots \text{ are open intervals and } A \subseteq \bigcup_{k=1}^\infty I_k\right\} = t|A| = |t||A|$. $\square$

\textbf{3} Prove that if $A, B \subset \R$ and $|A| < \infty$, then $|B - A| \geq |B| - |A|$.

\emph{Proof.} Observe that $B = (B - A) \cup (A \cap B)$. By subadditivity and order preservation we have $|B| - |A| = |(B - A) \cup (A \cap B)| - |A| \leq |B - A| + |A \cap B| - |A| = |B - A| - (|A| - |A \cap B|) \leq |B - A|$. $\square$

\textbf{4} Suppose $F$ is a subset of $\R$ with the property that every open cover of $F$ has a finite subcover. Prove that $F$ is closed and bounded.

\emph{Proof.} Obviously a set must be bounded in order to be compact. We now show that any non-closed set has an open set with no finite subcover. If a set $E$ is not closed, there is a limit point $c \in \R$ of $E$ such that $c \notin E$; particularly, given any radius $r > 0$, there exists a point $x \in E$ such that $0 < |x - c| < r$. Let $\{x_k\}$ be a sequence of points in $E$ such that $|x_{k+1} - c| \leq |x_k - c|$ for all $k \in \Z^+$ and $x_k \to x$ as $k \to \infty$. Let $\{I_k\}$ be a countable covering of $E$ by open intervals such that $x_i \in I_j$ if and only if $i = j$ and $x \notin \bigcup_{k=1}^\infty I_k$ (these criteria are possible because $\R$ is Hausdorff). $\{I_k\}$ is a ``minimal'' covering of $E$ in the sense that if any $I_k$ were removed, the sequence would no longer form a cover of $E$. As $\{I_k\}$ is an uncountable open cover, $E$ is not compact. $\square$

\textbf{5} Suppose $\mathcal{A}$ is a set of closed subsets of $\R$ such that $\bigcap_{F \in \mathcal{A}} F = \emptyset$. Prove that if $\mathcal{A}$ contains at least one bounded set, then there exist $n \in \Z^+$ and $F_1, \ldots, F_n \in \mathcal{A}$ such that $F_1 \cap \ldots \cap F_n = \emptyset$.

\emph{Proof.} Suppose $\mathcal{A}$ contains a bounded set $S$. Then
\begin{align}
\begin{split}
    S \cap \bigcap_{F \in \mathcal{A}} F &= \emptyset \\
    \implies S \subseteq \overline{\bigcap_{F \in \mathcal{A}} F} &= \bigcup_{F \in \mathcal{A}} \overline{F}.
\end{split}
\end{align}
By Heine-Borel theorem, $S$ is compact. Using this result, and because $\{\overline{F}\}_{F \in \mathcal{A}}$ is an open cover of $S$, there exists a finite subcover $\{\overline{F_k}\}_{k=2}^n$ of $S$. Hence
\begin{align}
\begin{split}
    S \subseteq \bigcup_{k=2}^n \overline{F_k} &= \overline{\bigcap_{k=2}^n F_k} \\
    \implies S \cap \bigcap_{k=2}^n F_k &= \emptyset.
\end{split}
\end{align}
Letting $F_1 = S$, we have attained the desired result. $\square$

\textbf{6} Prove that if $a, b \in \R$ and $a < b$, then $|(a, b)| = |[a, b)| = |(a, b]| = b - a$.

\emph{Proof.} Let $\varepsilon > 0$ satisfy $2\varepsilon \leq b - a$ but otherwise be chosen arbitrarily. Note that by the order preservation of outer measure we have $b - a - 2\varepsilon = |[a + \varepsilon, b - \varepsilon]| \leq |I| \leq |[a, b]| = b - a$ for $I = (a, b), I = [a, b),$ and $I = (a, b]$. Concisely, we have $b - a - 2\varepsilon \leq |I| \leq b - a$. Since $\varepsilon$ may be arbitrarily small, we may take the limit: $\displaystyle \lim_{\varepsilon \to 0^+} b - a - 2\varepsilon \leq \lim_{\varepsilon \to 0^+} |I| \leq \lim_{\varepsilon \to 0^+} b - a$. Thus $b - a \leq |I| \leq b - a$ so that $|I| = b - a$. $\square$

\textbf{7} Suppose $a, b, c, d$ are real numbers with $a < b$ and $c < d$. Prove that $|(a, b) \cup (c, d)| = (b - a) + (d - c)$ if and only if $(a, b) \cap (c, d) = \emptyset$.

\emph{Proof.} By subadditivity we know that $|(a, b) \cup (c, d)| \leq (b - a) + (d - c)$.

$(\impliedby)$ Suppose that $(a, b) \cap (c, d) = \emptyset$. Then
\begin{align}
\begin{split}
    |(a, b) \cap (c, d)| &= \inf\left\{\left.\sum_{k=1}^\infty |I_k| \right| I_1, I_2, \ldots \text{ are open intervals and } (a, b) \cap (c, d) \subseteq \bigcup_{k=1}^\infty I_k\right\} \\ \ \\
    &\geq \inf\left\{\left.\sum_{k=1}^\infty |I_k| \right| \ldots \text{and } (a, b) \subseteq \bigcup_{k=1}^\infty I_k\right\} + \inf\left\{\left.\sum_{k=1}^\infty +I_k| \right| \ldots \text{and } (c, d) \subseteq \bigcup_{k=1}^\infty I_k\right\} \\ \ \\
    &= |(a, b)| + |(c, d)| = (b - a) + (d - c).
\end{split}
\end{align}

$(\implies)$ On the other hand, suppose that $(a, b) \cap (c, d) = (c, b) \neq \emptyset$. Then $(a, b) \cup (c, d) = (a, d)$ so that $|(a, b) \cup (c, d)| = d - a < (d - a) + (b - c) = (b - a) + (d - c)$. $\square$

\textbf{8} Prove that if $A \subset \R$ and $t > 0$, then $|A| = |A \cap (-t, t)| + |A \cap (\R - (-t, t))|$.

\emph{Proof.} In general, subdivision of a set by intersection with finitely many disjoint intervals will satisfy additivity of outer measure. To show this for the problem at hand, write
\begin{align}
\begin{split}
    |A| &= \inf\left\{\left.\sum_{k=1}^\infty |I_k| \right| I_1, I_2, \ldots \text{ are open intervals and } A \subseteq \bigcup_{k=1}^\infty I_k\right\} \\ \ \\
    &= \inf\left\{\left.\sum_{k=1}^\infty (|I_k \cap (-\infty, -t)| + |I_k \cap (-t, t)| + |I_k \cap (t, \infty)|) \right| \ldots \text{and } A \subseteq \bigcup_{k=1}^\infty I_k\right\}.
\end{split}
\end{align}
To see that additivity holds here, observe that if any interval $I_k$ has $t$ (or $-t$) in its interior, the outer measure of that interval is the same as the sum of outer measures $I_k \cap (-\infty, t)$ and $I_k \cap (t, \infty)$ (or the sum of $I_k \cap (-\infty, -t)$ and $I_k \cap (-t, \infty)$). Thus
\begin{align}
\begin{split}
    |A| &= \inf\left\{\left.\sum_{k=1}^\infty |I_k \cap (-\infty, -t)| \right| \ldots \text{and } A \subseteq \bigcup_{k=1}^\infty I_k\right\} \\ \ \\
    &+ \inf\left\{\left.\sum_{k=1}^\infty |I_k \cap (-t, t)| \right| \ldots \text{and } A \subseteq \bigcup_{k=1}^\infty I_k\right\} \\ \ \\
    &+ \inf\left\{\left.\sum_{k=1}^\infty |I_k \cap (t, \infty)| \right| \ldots \text{and } A \subseteq \bigcup_{k=1}^\infty I_k\right\} \\ \ \\
    &= |A \cap (-t, t)| + |A \cap (-\infty, -t)| + |A \cap (t, \infty)|.
\end{split}
\end{align}
By the same logic, we have
\begin{align}
\begin{split}
    |A \cap (-t, t)| + |A \cap \overline{(-t, t)}| &= |A \cap (-t, t)| + |A \cap ((-\infty, -t) \cup (t, \infty))| \\ \ \\
    &= |A \cap (-t, t)| + |A \cap (-\infty, -t)| + |A \cap (t, \infty)|.
\end{split}
\end{align}
Hence $|A| = |A \cap (-t, t)| + |A \cap \overline{(-t, t)}|$. $\square$

\textbf{9} Prove that $|A| = \lim_{t \to \infty} |A \cap (-t, t)|$ for all $A \subset \R$.

\emph{Proof.} Suppose that $|A| < \infty$. Then, for any $\varepsilon > 0$, there exists a cover of $A$ by open intervals such that $\displaystyle \sum_{k=1}^\infty |I_k| = |A| + \varepsilon$. That is, $\displaystyle \lim_{n \to \infty} \sum_{k=1}^n |I_k| = |A| + \varepsilon$. Looking a little more carefully, for any $\varepsilon' > 0$, there exists $M \in \Z^+$ such that $n \geq M$ implies $\displaystyle \left(\sum_{k=1}^n |I_k|\right) - |A| - \varepsilon < \varepsilon'$, or that $\displaystyle \left(\sum_{k=1}^n |I_k|\right) - |A| < \varepsilon + \varepsilon'$. Essentially, no matter what ordering is chosen on the sequence of intervals, we can find a finite subsequence that has total measure arbitrarily close to $|A|$. Of course, the corresponding union $\bigcup_{k=1}^n I_k$ is a bounded set. Fixing $\varepsilon + \varepsilon'$, we can always find a bound $t$ so that $\bigcup_{k=1}^n I_k \subseteq (-t, t)$, which yields the desired result. $\square$

\textbf{10} Prove that $|[0, 1] - \Q| = 1$.

\textbf{11} Prove that if $I_1, I_2, \ldots$ is a disjoint sequence of open intervals, then $\left|\bigcup_{k=1}^\infty I_k\right| = \sum_{k=1}^\infty m(I_k)$.

\textbf{12} Suppose $r_1, r_2, \ldots$ is a sequence that contains every rational number. Let $F = \R - \bigcup_{k=1}^\infty \left(r_k - \frac{1}{2^k}, r_k + \frac{1}{2^k}\right)$.

(a) Show that $F$ is a closed subset of $\R$.

(b) Prove that if $I$ is an interval contained in $F$, then $I$ contains at most one element.

(c) Prove that $|F| = \infty$.

\textbf{13} Suppose $\varepsilon > 0$. Prove that there exists a subset $F$ of $[0, 1]$ such that $F$ is closed, every element of $F$ is an irrational number, and $|F| > 1 - \varepsilon$.

\end{document}